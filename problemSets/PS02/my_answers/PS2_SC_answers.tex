\documentclass[12pt,letterpaper]{article}
\usepackage{graphicx,textcomp}
\usepackage{natbib}
\usepackage{setspace}
\usepackage{fullpage}
\usepackage{color}
\usepackage[reqno]{amsmath}
\usepackage{amsthm}
\usepackage{fancyvrb}
\usepackage{amssymb,enumerate}
\usepackage[all]{xy}
\usepackage{endnotes}
\usepackage{lscape}
\newtheorem{com}{Comment}
\usepackage{float}
\usepackage{hyperref}
\newtheorem{lem} {Lemma}
\newtheorem{prop}{Proposition}
\newtheorem{thm}{Theorem}
\newtheorem{defn}{Definition}
\newtheorem{cor}{Corollary}
\newtheorem{obs}{Observation}
\usepackage[compact]{titlesec}
\usepackage{dcolumn}
\usepackage{tikz}
\usetikzlibrary{arrows}
\usepackage{multirow}
\usepackage{xcolor}
\newcolumntype{.}{D{.}{.}{-1}}
\newcolumntype{d}[1]{D{.}{.}{#1}}
\definecolor{light-gray}{gray}{0.65}
\usepackage{url}
\usepackage{listings}
\usepackage{color}
\usepackage{booktabs}

\definecolor{codegreen}{rgb}{0,0.6,0}
\definecolor{codegray}{rgb}{0.5,0.5,0.5}
\definecolor{codepurple}{rgb}{0.58,0,0.82}
\definecolor{backcolour}{rgb}{0.95,0.95,0.92}

\lstdefinestyle{mystyle}{
	backgroundcolor=\color{backcolour},   
	commentstyle=\color{codegreen},
	keywordstyle=\color{magenta},
	numberstyle=\tiny\color{codegray},
	stringstyle=\color{codepurple},
	basicstyle=\footnotesize,
	breakatwhitespace=false,         
	breaklines=true,                 
	captionpos=b,                    
	keepspaces=true,                 
	numbers=left,                    
	numbersep=5pt,                  
	showspaces=false,                
	showstringspaces=false,
	showtabs=false,                  
	tabsize=2
}
\lstset{style=mystyle}
\newcommand{\Sref}[1]{Section~\ref{#1}}
\newtheorem{hyp}{Hypothesis}

\title{Problem Set 2}
\date{Due: February 18, 2024}
\author{Sara Cid}


\begin{document}
	\maketitle
	\section*{Instructions}
	\begin{itemize}
		\item Please show your work! You may lose points by simply writing in the answer. If the problem requires you to execute commands in \texttt{R}, please include the code you used to get your answers. Please also include the \texttt{.R} file that contains your code. If you are not sure if work needs to be shown for a particular problem, please ask.
		\item Your homework should be submitted electronically on GitHub in \texttt{.pdf} form.
		\item This problem set is due before 23:59 on Sunday February 18, 2024. No late assignments will be accepted.
	%	\item Total available points for this homework is 80.
	\end{itemize}

	%	\vspace{.25cm}

	\vspace{.25cm}
%\section*{Question 1} %(20 points)}
%\vspace{.25cm}
\noindent We're interested in what types of international environmental agreements or policies people support (\href{https://www.pnas.org/content/110/34/13763}{Bechtel and Scheve 2013)}. So, we asked 8,500 individuals whether they support a given policy, and for each participant, we vary the (1) number of countries that participate in the international agreement and (2) sanctions for not following the agreement. \\

\noindent Load in the data labeled \texttt{climateSupport.RData} on GitHub, which contains an observational study of 8,500 observations.

\begin{itemize}
	\item
	Response variable: 
	\begin{itemize}
		\item \texttt{choice}: 1 if the individual agreed with the policy; 0 if the individual did not support the policy
	\end{itemize}
	\item
	Explanatory variables: 
	\begin{itemize}
		\item
		\texttt{countries}: Number of participating countries [20 of 192; 80 of 192; 160 of 192]
		\item
		\texttt{sanctions}: Sanctions for missing emission reduction targets [None, 5\%, 15\%, and 20\% of the monthly household costs given 2\% GDP growth]
		
	\end{itemize}
	
\end{itemize}

\newpage
\noindent Please answer the following questions:

\begin{enumerate}
	\item
	Remember, we are interested in predicting the likelihood of an individual supporting a policy based on the number of countries participating and the possible sanctions for non-compliance.
	\begin{enumerate}
		\item [] Fit an additive model. Provide the summary output, the global null hypothesis, and $p$-value. Please describe the results and provide a conclusion
	\end{enumerate}
	
	\vspace{.25cm}
	\noindent \textbf{Answer to Question 1:}
	\vspace{.25cm}
	
	\noindent In this part, we fit a logistic regression where we examine the influence of the number of participating countries and the level of sanctions on individual support for policies. The global null hypothesis we are testing, $H_0$, is that both variables do not significantly affect policy support. This can be expressed as: $H_0: \beta_{\text{countries}} = \beta_{\text{sanctions}} = 0$. As such, the alternative hypothesis $H_A$ is that least one of these predictors significantly impacts support likelihood. When fitting the regression model, we are also performing hypothesis tests for all the individual coefficients, using Wald tests. 
	
	First, we run the regression, making sure we set our preferred reference categories (in this case, `None' for sanctions and `20/192 countries', which is the baseline for country support): 
	\vspace{.25cm}
	
	\lstinputlisting[language=R, linerange={44-61}]{PS2_SC_answers.R}
	
	\newpage
	\noindent Next we perform a likelihood ratio test (LRT) to evaluate the global null hypothesis: 
	\vspace{.25cm}
	
	\lstinputlisting[language=R, linerange={78-85}]{PS2_SC_answers.R}
	\vspace{.25cm}
	
	\noindent Showing the results: 
	
	\begin{table}[!htbp] \centering   \caption{Summary of Null Logistic Regression Model}   \label{} \begin{tabular}{@{\extracolsep{5pt}}lc} \\[-1.8ex]\hline \hline \\[-1.8ex]  & \multicolumn{1}{c}{\textit{Dependent variable:}} \\ \cline{2-2} \\[-1.8ex] & choice \\ \hline \\[-1.8ex]  Constant & $-$0.007 \\   & (0.022) \\   & \\ \hline \\[-1.8ex] Observations & 8,500 \\ Log Likelihood & $-$5,891.705 \\ Akaike Inf. Crit. & 11,785.410 \\ \hline \hline \\[-1.8ex] \textit{Note:}  & \multicolumn{1}{r}{$^{*}$p$<$0.1; $^{**}$p$<$0.05; $^{***}$p$<$0.01} \\ \end{tabular} \end{table} 
	
	\begin{table}[!htbp] \centering   \caption{LRT Results}   \label{} \begin{tabular}{@{\extracolsep{5pt}}lccccc} \\[-1.8ex]\hline \hline \\[-1.8ex] Statistic & \multicolumn{1}{c}{N} & \multicolumn{1}{c}{Mean} & \multicolumn{1}{c}{St. Dev.} & \multicolumn{1}{c}{Min} & \multicolumn{1}{c}{Max} \\ \hline \\[-1.8ex] Resid. Df & 2 & 8,496.500 & 3.536 & 8,494 & 8,499 \\ Resid. Dev & 2 & 11,675.830 & 152.134 & 11,568.260 & 11,783.410 \\ Df & 1 & 5.000 &  & 5 & 5 \\ Deviance & 1 & 215.150 &  & 215.150 & 215.150 \\ Pr(\textgreater Chi) & 1 & 0.000 &  & 0 & 0 \\ \hline \\[-1.8ex] \end{tabular} \end{table} 
	
	\noindent From the results, we see that the p-value is smaller than 0.000. This means we can reject the global null hypothesis that all of the slopes are equal to zero, and we have find evidence for the alternative hypothesis that at least one of them is not. 
	
	\noindent Then we can look at individual coefficients. Table 1 shows the coefficient estimates, p-values, and the number of iterations it took to find the maximum likelihood estimate: 
	
		\begin{table}[!htbp] \centering   \caption{Model Summary}   \label{} \begin{tabular}{@{\extracolsep{5pt}}lc} \\[-1.8ex]\hline \hline \\[-1.8ex]  & \multicolumn{1}{c}{\textit{Dependent variable:}} \\ \cline{2-2} \\[-1.8ex] & choice \\ \hline \\[-1.8ex]  countries80 of 192 & 0.336$^{***}$ \\   & (0.054) \\   & \\  countries160 of 192 & 0.648$^{***}$ \\   & (0.054) \\   & \\  sanctions5\% & 0.192$^{***}$ \\   & (0.062) \\   & \\  sanctions15\% & $-$0.133$^{**}$ \\   & (0.062) \\   & \\  sanctions20\% & $-$0.304$^{***}$ \\   & (0.062) \\   & \\  Constant & $-$0.273$^{***}$ \\   & (0.054) \\   & \\ \hline \\[-1.8ex] Number of Fisher Scoring iterations & 4 \\ Observations & 8,500 \\ Log Likelihood & $-$5,784.130 \\ Akaike Inf. Crit. & 11,580.260 \\ \hline \hline \\[-1.8ex] \textit{Note:}  & \multicolumn{1}{r}{$^{*}$p$<$0.1; $^{**}$p$<$0.05; $^{***}$p$<$0.01} \\ \end{tabular} \end{table} 
		
	\newpage
	\noindent Interpreting the coefficients from our logistic regression model:
	\begin{itemize}
		\item countries80 of 192: This coefficient of 0.336 means that moving from the reference category of 20 out of 192 countries to the one of 80 out of 192 countries participating increases the log odds of supporting the policy by 0.336, on average, while controlling for sanctions. 
		\item countries160 of 192: This coefficient of 0.648 indicates that moving from 20 out of 192 countries to 160 out of 192 countries increases the log odds of support by 0.648, on average, while controlling for sanctions. 
		\item sanctions5\%: The coefficient of 0.192 for 5\% sanctions means that introducing 5\% sanctions, compared to none, is associated with an increase of 0.192 in the log odds, on average, while controlling for the number of participating countries. 
		\item sanctions15\%: This coefficient of -0.133 for 15\% sanctions means that increasing sanctions from none to this level is associated with a decrease of -0.133 in the log odds, on average, while controlling for the number of participating countries. 
		\item sanctions20\%: This coefficient of -0.304 for 20\% sanctions means that increasing sanctions from none to this level is associated with a decrease of -0.304 in the log odds, on average, while controlling for the number of participating countries. This decrease is more substantive than the one going from no sanctions to 15\%. 
		\item Constant (Intercept): Finally, the intercept of -0.273 means that the log odds of supporting the policy when all of our independent variables are at their reference levels (no sanctions and 20 out of 192 countries participating) is -0.273, on average. 
		\item Moreover, all coefficients are statistically significant, and the corresponding p-values are always less than 0.01. Thus, for each of the coefficients, we can reject the null hypothesis that the coefficient is equal to zero. 
	\end{itemize}
	
	\newpage
	\noindent Additionally, we can "transform" our results to make them more informative on the relationship between changing from the different categories of our independent variables and both the odds of supporting a policy and the probability of supporting a policy:
	\vspace{.25cm} 	
	
	\lstinputlisting[language=R, linerange={97-128}]{PS2_SC_answers.R}
	
	\newpage
	\noindent Showing the results: 
	
		\begin{table}[!htbp] \centering   \caption{Effects of Countries Participation and Sanctions on Policy Support}   \label{} \begin{tabular}{@{\extracolsep{5pt}} cccc} \\[-1.8ex]\hline \hline \\[-1.8ex]  & Variable & Odds & Probability \\ \hline \\[-1.8ex] 1 & 80 of 192 countries & $1.399$ & $0.583$ \\ 2 & 160 of 192 countries & $1.912$ & $0.657$ \\ 3 & Sanctions 5\% & $1.212$ & $0.548$ \\ 4 & Sanctions 15\% & $0.875$ & $0.467$ \\ 5 & Sanctions 20\% & $0.738$ & $0.425$ \\ \hline \\[-1.8ex] \end{tabular} \end{table} 
	
	\noindent Interpretation: 
	
	\noindent `Effects on Odds'
	\begin{itemize}
		\item 80 of 192 countries: The odds of supporting the policy increase by 1.399, the factor shown for `80 of 192 countries' compared to the baseline, meaning the odds are increased by around 40\%, on average, while controlling for sanctions. 
		\item 160 of 192 countries: For `160 of 192 countries"', the odds of support increase by a factor of 1.912, or by 91.2\%, on average, while controlling for sanctions. 
		\item Sanctions 5\%: Introducing sanctions at a 5\% level increases the odds of policy support by a factor of 1.212, or 21.1\%, on average, while controlling for participating countries. 
		\item Sanctions 15\%:  Introducing sanctions at a 15\% level decreases the odds of policy support by a factor of 0.875, or 12.5\%, on average, while controlling for participating countries. 
		\item Sanctions 20\%:  Introducing sanctions at a 20\% level increases the odds of policy support by a factor of 0.738, or 26.2\%, on average, while controlling for participating countries. 
	\end{itemize}
	
	\noindent `Effects on Probabilities'
	\begin{itemize}
		\item 80 of 192 countries: Changing from 20 to 80 countries out of 192 is associated with an increase of 0.583 in the probability of support, on average,  while controlling for sanctions. 
		\item 160 of 192 countries: Changing from 20 to 80 countries out of 192 is associated with an increase of 0.657 in the probability of support, on average,  while controlling for sanctions. 
		\item Sanctions 5\%: Changing from no sanctions to 5\% is associated with an increase of 0.548 in the probability of support, on average, while controlling for participating countries. 
		\item Sanctions 15\% Changing from no sanctions to 15\% is associated with a decrease of 0.467 in the probability of support, on average, while controlling for participating countries. 
		\item Sanctions 20\%: Changing from no sanctions to 20\% is associated with a decrease of 0.425 in the probability of support, on average, while controlling for participating countries.  
	\end{itemize}
	
	\newpage
	\item
	If any of the explanatory variables are significant in this model, then:
	\begin{enumerate}
		\item
		For the policy in which nearly all countries participate [160 of 192], how does increasing sanctions from 5\% to 15\% change the odds that an individual will support the policy? (Interpretation of a coefficient)
		
		\vspace{.25cm} 	
		\textbf{Answer to Question 2A}
		\vspace{.25cm} 	
		
		\noindent Changing the reference category to 5\% to be able to make a direct interpretation and showing the results of fitting the model again: 
		
		\vspace{.25cm} 	
		\lstinputlisting[language=R, linerange={137-145}]{PS2_SC_answers.R}
		\vspace{.25cm} 	
		
		\begin{table}[!htbp] \centering   \caption{Model Summary}   \label{} \begin{tabular}{@{\extracolsep{5pt}}lc} \\[-1.8ex]\hline \hline \\[-1.8ex]  & \multicolumn{1}{c}{\textit{Dependent variable:}} \\ \cline{2-2} \\[-1.8ex] & choice \\ \hline \\[-1.8ex]  countries80 of 192 & 0.336$^{***}$ \\   & p = 0.000 \\   & \\  countries160 of 192 & 0.648$^{***}$ \\   & p = 0.000 \\   & \\  sanctions2None & $-$0.192$^{***}$ \\   & p = 0.003 \\   & \\  sanctions215\% & $-$0.325$^{***}$ \\   & p = 0.00000 \\   & \\  sanctions220\% & $-$0.495$^{***}$ \\   & p = 0.000 \\   & \\  Constant & $-$0.081 \\   & p = 0.129 \\   & \\ \hline \\[-1.8ex] Number of Fisher Scoring iterations & 4 \\ Observations & 8,500 \\ Log Likelihood & $-$5,784.130 \\ Akaike Inf. Crit. & 11,580.260 \\ \hline \hline \\[-1.8ex] \textit{Note:}  & \multicolumn{1}{r}{$^{*}$p$<$0.1; $^{**}$p$<$0.05; $^{***}$p$<$0.01} \\ \end{tabular} \end{table} 
		
		\vspace{.25cm} 	
		\noindent As we can see, the coefficient for moving from 5\% (reference category) to 15\% is -0.325, so when making this change in sanctions (from 5\% to 15\%), we expect the log odds of supporting a policy to decrease by 0.325. We can also "translate" this so we get information about what this change in categories would do to the odds and the probability: 
		
		\vspace{.25cm} 	
		\lstinputlisting[language=R, linerange={162-170}]{PS2_SC_answers.R}
		\vspace{.25cm} 	
		
		The results of "transforming" are that, for increasing sanctions from 5\% to 15\%, the odds ratio is \textbf{0.722}. This means there is a decrease in the OR by approximately 28\% when sanctions are increased from 5\% to 15\%, all else in the model constant. In terms of the probability of support, this probability under 15\% sanctions, compared to 5\%, is \textbf{0.4194333} or approximately 42\%
		
		\newpage
		\item
		What is the estimated probability that an individual will support a policy if there are 80 of 192 countries participating with no sanctions? 
		
		\vspace{.25cm} 	
		\textbf{Answer to Question 2B}
		\vspace{.25cm} 	
		
		We are asked to see what the probability of support is when Sanctions is `None' and Countries is '180 out of 192`. We can calculate this as follows: 
		
		\vspace{.25cm} 	
		\lstinputlisting[language=R, linerange={173-196}]{PS2_SC_answers.R}
		\vspace{.25cm} 	
		
		\noindent We see that the estimated probability is 0.515, or 51.5\%, which indicates that when 80 out of 192 countries participate in a policy without implementing any sanctions, there is approximately a fifty-fifty chance that an individual will support it. 
		
		\newpage
		\item
		Would the answers to 2a and 2b potentially change if we included the interaction term in this model? Why? 
		\begin{itemize}
			\item Perform a test to see if including an interaction is appropriate.
			
		\vspace{.25cm} 		
		\textbf{Answer to Question 2C}
		\vspace{.25cm} 	
		
		\noindent The results could change, because including an interaction between sanctions and countries would allow for the effect of one variable (for instance, countries) on the outcome to vary according to the other variable (for instance, sanction). To evaluate whether including an interaction is appropriate, we can fit a model with the interaction and evaluate its performance compared to the additive one, by performing another likelihood ratio test: 
		
		\vspace{.25cm} 	
		\lstinputlisting[language=R, linerange={199-211}]{PS2_SC_answers.R}
		\vspace{.25cm} 	
	
		\noindent Looking at the results from the test, and at all the estimated coefficients for the interactive model: 
		
		\begin{table}[!htbp] \centering   \caption{LRT Results}   \label{} \begin{tabular}{@{\extracolsep{5pt}}lccccc} \\[-1.8ex]\hline \hline \\[-1.8ex] Statistic & \multicolumn{1}{c}{N} & \multicolumn{1}{c}{Mean} & \multicolumn{1}{c}{St. Dev.} & \multicolumn{1}{c}{Min} & \multicolumn{1}{c}{Max} \\ \hline \\[-1.8ex] Resid. Df & 2 & 8,491.000 & 4.243 & 8,488 & 8,494 \\ Resid. Dev & 2 & 11,565.110 & 4.450 & 11,561.970 & 11,568.260 \\ Df & 1 & 6.000 &  & 6 & 6 \\ Deviance & 1 & 6.293 &  & 6.293 & 6.293 \\ Pr(\textgreater Chi) & 1 & 0.391 &  & 0.391 & 0.391 \\ \hline \\[-1.8ex] \end{tabular} \end{table} 
	
		\noindent From the LRT, we get a p-value of 0.3912, so we are not able to reject the null hypothesis, and the additive model performs better than the interactive one. Additionally, looking at all the estimated coefficients, we can see that the answers to questions 2a and 2b do change when including an interaction, as the estimated coefficients are different in the interactive model compared to the additive one. 
		
		\begin{table}[!htbp] \centering   \caption{Summary of Null Logistic Regression Model}   \label{} \footnotesize \begin{tabular}{@{\extracolsep{5pt}}lc} \\[-1.8ex]\hline \hline \\[-1.8ex]  & \multicolumn{1}{c}{\textit{Dependent variable:}} \\ \cline{2-2} \\[-1.8ex] & choice \\ \hline \\[-1.8ex]  countries80 of 192 & 0.376$^{***}$ \\   & (0.106) \\   & \\  countries160 of 192 & 0.613$^{***}$ \\   & (0.108) \\   & \\  sanctions5\% & 0.122 \\   & (0.105) \\   & \\  sanctions15\% & $-$0.097 \\   & (0.108) \\   & \\  sanctions20\% & $-$0.253$^{**}$ \\   & (0.108) \\   & \\  countries80 of 192:sanctions5\% & 0.095 \\   & (0.152) \\   & \\  countries160 of 192:sanctions5\% & 0.130 \\   & (0.151) \\   & \\  countries80 of 192:sanctions15\% & $-$0.052 \\   & (0.152) \\   & \\  countries160 of 192:sanctions15\% & $-$0.052 \\   & (0.153) \\   & \\  countries80 of 192:sanctions20\% & $-$0.197 \\   & (0.151) \\   & \\  countries160 of 192:sanctions20\% & 0.057 \\   & (0.154) \\   & \\  Constant & $-$0.275$^{***}$ \\   & (0.075) \\   & \\ \hline \\[-1.8ex] Observations & 8,500 \\ Log Likelihood & $-$5,780.983 \\ Akaike Inf. Crit. & 11,585.970 \\ \hline \hline \\[-1.8ex] \textit{Note:}  & \multicolumn{1}{r}{$^{*}$p$<$0.1; $^{**}$p$<$0.05; $^{***}$p$<$0.01} \\ \end{tabular} \end{table} 
		
		\newpage
		\noindent This is clearer when looking at the results of both models side by side: 
		
		\begin{table}[!htbp] \centering   \caption{Model Summary}   \label{} \footnotesize \begin{tabular}{@{\extracolsep{3pt}}lcc} \\[-1.8ex]\hline \hline \\[-1.8ex]  & \multicolumn{2}{c}{\textit{Dependent variable:}} \\ \cline{2-3} \\[-1.8ex] & \multicolumn{2}{c}{choice} \\ \\[-1.8ex] & (1) & (2)\\ \hline \\[-1.8ex]  countries80 of 192 & 0.336$^{***}$ & 0.376$^{***}$ \\   & p = 0.000 & p = 0.0005 \\   & & \\  countries160 of 192 & 0.648$^{***}$ & 0.613$^{***}$ \\   & p = 0.000 & p = 0.000 \\   & & \\  sanctions5\% & 0.192$^{***}$ & 0.122 \\   & p = 0.003 & p = 0.247 \\   & & \\  sanctions15\% & $-$0.133$^{**}$ & $-$0.097 \\   & p = 0.032 & p = 0.371 \\   & & \\  sanctions20\% & $-$0.304$^{***}$ & $-$0.253$^{**}$ \\   & p = 0.00001 & p = 0.020 \\   & & \\  countries80 of 192:sanctions5\% &  & 0.095 \\   &  & p = 0.535 \\   & & \\  countries160 of 192:sanctions5\% &  & 0.130 \\   &  & p = 0.390 \\   & & \\  countries80 of 192:sanctions15\% &  & $-$0.052 \\   &  & p = 0.731 \\   & & \\  countries160 of 192:sanctions15\% &  & $-$0.052 \\   &  & p = 0.736 \\   & & \\  countries80 of 192:sanctions20\% &  & $-$0.197 \\   &  & p = 0.192 \\   & & \\  countries160 of 192:sanctions20\% &  & 0.057 \\   &  & p = 0.712 \\   & & \\  Constant & $-$0.273$^{***}$ & $-$0.275$^{***}$ \\   & p = 0.00000 & p = 0.0003 \\   & & \\ \hline \\[-1.8ex] Number of Fisher Scoring iterations & 4 &  \\ Observations & 8,500 & 8,500 \\ Log Likelihood & $-$5,784.130 & $-$5,780.983 \\ Akaike Inf. Crit. & 11,580.260 & 11,585.970 \\ \hline \hline \\[-1.8ex] \textit{Note:}  & \multicolumn{2}{r}{$^{*}$p$<$0.1; $^{**}$p$<$0.05; $^{***}$p$<$0.01} \\ \end{tabular} \end{table}
		
				
		\end{itemize}
	\end{enumerate}
	\end{enumerate}


\end{document}
